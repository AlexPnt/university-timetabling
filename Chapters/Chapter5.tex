% Chapter 5

\chapter{Conclusions} % Main chapter title
%\epigraph{A fancy quote.}{Me}

\label{conclusion} % For referencing the chapter elsewhere, use \ref{Chapter2} 

\lhead{Chapter 5. \emph{Conclusions}} % This is for the header on each page - perhaps a shortened title

In this work, an example of a timetabling solver is presented. The proposed algorithm is based on the iterated local search framework, a technique that usually performs well in many combinatorial problems. \\
We started by defining the problem, in Chapter \ref{current_process}, by giving a detailed description of its main components, i.e, by describing important concepts and first introducing the concept of constraints.\\
In Chapter \ref{art} we made a survey of a wide range of existing techniques to solve the timetabling problem and also presented a technological review of existing software implementations that currently solve timetabling problems.\\
The last chapter described in great detail the data structures used and the complexity involved in common operations such as evaluating constraint violations. It also described the nature of the objective function, i.e, how for a given a solution, a value representative of its quality should be achieved. Finally in the last section, an overview of an example algorithm was described, namely how an initial feasible solution could be achieved, what neighbourhood structures could be used and finally the iterated local search based algorithm.\par
